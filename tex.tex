\documentclass[12pt]{article}
%\usepackage{natbib}
\usepackage[french]{babel}
\usepackage{url}
\usepackage[utf8x]{inputenc}
\usepackage{graphicx}
\graphicspath{{images/}}
\usepackage{parskip}
\usepackage{fancyhdr}
\usepackage{vmargin}
\usepackage{xcolor}
\usepackage{bbm}
\usepackage{amsmath,amssymb}
\usepackage{amsthm}
\usepackage{dsfont}
\usepackage{stmaryrd}
\usepackage{systeme}
\usepackage{enumitem}
\usepackage{xcolor}
\usepackage{pifont}
\usepackage{textcomp}

\title{Nombres algébriques et résultant}
\author{CARVAILLO T.}
\date{\today}

\makeatletter
\let\thetitle\@title
\let\theauthor\@author
\let\thedate\@date
\makeatother

\pagestyle{fancy}
\fancyhf{}
\rhead{\theauthor}
\lhead{\thetitle}
\cfoot{\thepage}
\def\dotfill#1{\cleaders\hbox to #1{.}\hfill}
\newcommand\dotline[2][.5em]{\leavevmode\hbox to #2{\dotfill{#1}\hfil}}

\newcommand{\jL}{\mathbbm{L}}
\newcommand{\Z}{\mathbbm{Z}}
\newcommand{\Q}{\mathbbm{Q}}
\newcommand{\R}{\mathbbm{R}}
\newcommand{\C}{\mathbbm{C}}
\newcommand{\K}{\mathbbm{K}}
\newcommand{\F}{\mathbbm{F}}
\newcommand{\Fq}{\mathbbm{F}_q}
\newcommand{\Fqn}{\mathbbm{F}_{q^n}}

%définition commande présentation fonction
\newcommand{\fonction}[5]{
\begin{displaymath}
\begin{array}{l|rcl}
\displaystyle
#1 : & #2 & \longrightarrow & #3 \\
    & #4 & \longmapsto & #5
\end{array}
\end{displaymath}
}
%fin définition

% de jolies accolades
\newcommand{\accolade}[2]{
\begin{displaymath}
%#1 = \left\{
    \begin{array}{ll}
       #1 %& \mbox{si }
       #2 %& \mbox{sinon.}
    \end{array}
\right.
\end{displaymath}
}
% de jolies accolades




\newtheorem{prop}{Proposition}
\newtheorem{thm}{Théorème}
\newtheorem{cor}{Corollaire}
\newtheorem{lem}{Lemme}
\theoremstyle{definition}\newtheorem{defn}{Définition}
\theoremstyle{definition}\newtheorem{exm}{Exemple}
\theoremstyle{definition}\newtheorem{rem}{Remarque}
\theoremstyle{definition}\newtheorem{algo}{Algorithme}
\theoremstyle{remark}\newtheorem{exo}{Exercice}
\theoremstyle{remark}\newtheorem{nota}{Notation}


\begin{document}

%%%%%%%%%%%%%%%%%%%%%%%%%%%%%%%%%%%%%%%%%%%%%%%%%%%%%%%%%%%%%%%%%%%%%%%%%%%%%%%%%%%%%%%%%

\begin{titlepage}
	\centering
    \vspace*{0.5 cm}
    \textsc{\LARGE Projet de Calcul Formel}\\[1.0 cm]
	\vspace{1.5cm}
	\rule{\linewidth}{0.2 mm} \\[0.4 cm]
	{ \huge \bfseries  \thetitle}\\ %\color{blue}
	\rule{\linewidth}{0.2 mm} \\[1.5 cm]
	
	\begin{minipage}{0.4\textwidth}
		\begin{flushleft} \large
			\emph{A l'attention de :}\\
			NOM PROFS ??\\
			\phantom{a}\\
			\phantom{a}\\
		\end{flushleft}
	\end{minipage}
	\begin{minipage}{0.5\textwidth}
    	\begin{flushright} \large
		\emph{Rédigé par :}\\
		CARVAILLO Thomas \\
		JACQUET Raphael \\
		PIARD Arthur
		\end{flushright}
	\end{minipage}\\[2 cm]
\end{titlepage}

%%%%%%%%%%%%%%%%%%%%%%%%%%%%%%%%%%%%%%%%%%%%%%%%%%%%%%%%%%%%%%%%%%%%%%%%%%%%%%%%%%%%%%%%%

\tableofcontents

%\section*{\Huge Notation}
%\vspace{3cm}
%\begin{tabular}{p{4cm}p{15cm}}
%$\mathbbm{F}_q$ & Corps de \textit{Galois} à $q$ éléments.\\
%$\mathbbm{F}_q$* & Ensemble des éléments inversibles de $\mathbbm{F}_q$\\
%\end{tabular}
%\pagebreak

%%%%%%%%%%%%%%%%%%%%%%%%%%%%%%%%%%%%%%%%%%%%%%%%%%%%%%%%%%%%%%%%%%%%%%%%%%%%%%%%%%%%%%%%%
\section*{Introduction}
\addcontentsline{toc}{part}{Introduction}
\begin{figure}[h]
Intro
\end{figure}

\vfill \eject


%%%%%%%%%%%%%%%%%%%%%%%%%%%%%%%%%%%%%%%%%%%%%%%%%%%%%%%%%%%%%%%%%%%%%%%%%%%%%%%%%%%%%%%
\pagebreak 
%%%%%%%%%%%%%%%%%%%%%%%%%%%%%%%%%%%%%%%%%%%%%%%%%%%%%%%%%%%%%%%%%%%%%%%%%%%%%%%%%%%%%%%

\section{Un peu de théorie}


\subsection{Rappels}

\begin{defn}
On appelle corps tout anneau $A$ abélien unitaire dans lequel tout élément non nul est inversible, i.e. $A^{\times} = A \backslash\{0\}$.
\end{defn}

\begin{nota}
Dans ce qui suit, le corps de base sera noté $\K$ et désignera indifféremment, sauf indication contraire, $\Q$, $\R$ ou $\C$. 
\end{nota}

\begin{defn}
On appelle extension de $\K$ tout corps $\jL$ contenant un sous-corps isomorphe à $\K$. On notera $\jL / \K$ une telle extension.
\end{defn}

\begin{defn}
On appelle degré de l'extension $\jL / \K$ la dimension de $\jL$ en tant que $\K$-espace vectoriel. On le notera $[\jL : \K]$.
\end{defn}

\begin{prop}
multiplicativité du degré.
\end{prop}

\begin{defn}
On dit que $\jL / \K$ est finie si elle est de degré finie.
\end{defn}

\begin{prop}
L'ensemble $\K[X]$ des polynômes à coefficients dans $\K$ en l'indéterminée $X$ est muni d'une structure d'anneau Euclidien.
\end{prop}


\subsection{Eléments algébriques}

\begin{defn}
Soient $\jL/\K$ une extension de corps et $P(X) = \displaystyle\sum_{i=0}^{n}a_iX^i$ un polynôme de degré $n$ à coefficients dans $\K$. On considère le morphisme d'évaluation \fonction{ev_\alpha}{\K[X]}{\jL}{P(X)}{P(\alpha)}
Soit $I(\alpha) :=  ker(ev_\alpha) = \{ P \in \K[X]$ tels que $P(\alpha) = 0\}$; on a deux possibilites:

\begin{itemize}
	\item Soit $I(\alpha) \ne \{ 0 \}$, i.e. $ev_a$ n'est pas injective et donc $\exists P \in \K[X]\backslash\{0\}$ tel que $P(\alpha) = 0$. \newline
Dans ce cas $\alpha$ est dit algébrique sur $\K$.
	\item Soit $I(\alpha) = \{ 0 \}$ i.e. $ev_a$ est injective et donc $\nexists P \in \K[X]\backslash\{0\}$ tel que $P(\alpha) = 0$. \newline
Dans ce cas, $\alpha$ est dit transcendant sur $\K$.
\end{itemize}
\end{defn}


\begin{thm}
Soit $\jL / \K$ une extension de corps et $\alpha$ un élément algébrique sur $\jL$, alors il existe un unique polynôme $P(X)$ unitaire irréductible dans $\K[X]$ vérifiant
\begin{center} ($Q(X) \in \K[X]\backslash\{0\}$ et $Q(\alpha)=0$) ssi $P(X) \mid Q(X)$\end{center}
\end{thm}

\begin{proof}
$\K[X]$ est euclidien, donc en particulier principal. Il s'ensuit qu'il existe $P(X) \in \K[X]\backslash\{0\}$ unitaire tel que $I(a) = (P(X))$, $I(a)$ étant un idéal propre non nul. Par le premier théorème d'isomorphisme, on obtient que $Im(ev_\alpha) \simeq \frac{\K[X]}{(P(X))}$. \newline
Ce dernier étant intègre, on obtient que $P(X)$ est premier donc irréductible dans $\K[X]$ factoriel. \newline
Il s'ensuit naturellement que $Q(X) \in I(a)\backslash\{0\} = (P(X))\backslash\{0\}$ ssi $P(X) \mid Q(X)$.
\end{proof}

\begin{prop}[Admise]
On a de plus $deg(P) = [\jL : \K]$.
\end{prop}

\begin{defn}
Le polynôme $P(X)$ comme décrit ci-dessus est appellé le polynôme minimal de $\alpha$ sur $\K$ et est noté $Irr(\alpha,X,\K)$.
\end{defn}

\begin{rem}
Soit $\alpha \in \Q$, il peut être intéressant de remarquer qu'un polynôme irréductible dans $\Q[X]$ annulant $\alpha$ sera son toujours son polynôme minimal sur $\Q$. Cela découle de ce qui a été vu plus haut.
\end{rem}

\begin{prop}[Critère d'Eiseinstein - Admis]
Soit $P(X) = \displaystyle \sum_{i=0}^{n}a_iX^i$ un polynôme de $\Z[X]$, supposons de plus qu'il existe $p$ premier tel que $\forall i \in \llbracket 0, n-1 \rrbracket$
\begin{itemize}
	\item $p \mid a_i$, 
	\item $p \nmid a_n$
	\item $p^2 \nmid a_0$
\end{itemize}
alors $P(X)$ est irréductible dans $\Q[X]$.
\end{prop}

\begin{exm}
	\begin{itemize} Voyons quelques cas triviaux:
		\item $i$ est algébrique sur $\Q$, en effet $X^2-1$ est son polynôme minimal sur $Q$.
		\item $\sqrt2$ et $\sqrt3$ sont algébrique sur $\Q$, de polynôme minimaux respectif $X^2 -2$ et $X^2-3$, dont l'irréductibilité découle du critère d'Eisenstein.
		\item $\alpha = \sqrt2 + \sqrt3$ est également algébrique sur $\Q$. En effet, $\alpha = \sqrt2 + \sqrt3$ ssi $(\alpha-\sqrt2)^2= 3$ ssi $\alpha^2 + 2\alpha\sqrt2 + 2 = 3$ ssi $\alpha^2 - 1 =  -2\alpha\sqrt2$ ssi $\alpha^4 - 2\alpha^2 +1 =  8\alpha^2$ ssi $\alpha^4 - 10\alpha^2 +1 = 0$. $\alpha$ admet donc pour polynôme minimal $X^4 -10X^2 +1$. L'irréductibilité découle de Eisenstein pour $p=2$.	
	\end{itemize}
\end{exm}

\begin{defn}
Soit $\jL / \K$ une extension. On appelle fermeture algébrique de $\K$ dans $\jL$ l'ensemble des éléments de $\jL$ algébriques sur $\K$.
\end{defn}

\begin{defn}
On dit que $\jL / \K$ est algébrique si tout élément de $\jL$ est algébrique sur $\K$.
\end{defn}

\begin{prop}[Admise]
Une extension finie est algébrique.
\end{prop}

\begin{nota}
On notera $\K(\alpha_1, ..., \alpha_n)$ le plus petit corps, au sens de l'inclusion, contenant $\K$, $\alpha_1$, ..., $\alpha_n$.
\end{nota}

\begin{thm}
Soit $\jL/\K$ une extension de corps et soient $\alpha$ et $\beta$ deux éléments de $\jL$ non nuls algébriques sur $\K$. Alors, $\alpha + \beta$, $\alpha.\beta$ et $\alpha^{-1}$ sont algébriques sur $\K$. En d'autres termes, la fermeture algébrique de $\K$ est une extension de $\K$.
\end{thm}

\begin{proof}
Nous allons donner ici une première preuve non constructive. $\K(\alpha)/\K$ et $\K(\beta)/\K$ sont finies et $[\K(\alpha, \beta) : \K] = [\K(\alpha, \beta) : \K(\alpha)].[\K(\alpha) : \K]$ \newline
De plus, on a $K\subseteq \K(\alpha) \subseteq \K(\alpha,\beta)$ et $ \K\subseteq \K(\beta) \subseteq \K(\alpha,\beta)$ donc 
\begin{center}
$deg(Irr(\beta,X,\K(\alpha))) \le deg(Irr(\beta,X,\K))$
\end{center}
d'où 
\begin{center}
$[\K(\alpha,\beta) : \K] \le [\K(\beta) : \K].[\K(\alpha) : \K] < \infty$
\end{center}
Donc  $[\K(\alpha, \beta) : \K]$ est fini et l'extension est algébrique. Il s'ensuit naturellement que $\alpha + \beta$, $\alpha.\beta$ et $\alpha^{-1}$ sont algébriques, car contenus dans $\K(\alpha, \beta)$.
\end{proof}


\subsection{Résultants}

Introduisons maintenant une notion fondamentale, celle de \textit{résultant}, qui va nous permettre de donner une seconde démonstration - cette fois ci constructrice - du dernier théorème.

\begin{defn}
Soient $A = \displaystyle\sum_{i=0}^n a_iX^i$ et $B = \displaystyle\sum_{i=0}^m b_iX^i$ deux polynômes de $\K[X]$. On appelle matrice de Sylvester de $P$ et $Q$ la matrice de taille $(m+n)\times(m+n)$ définit par :
\begin{center}
$$\begin{array}{cc} 
Syl(A,B) :=
   \begin{pmatrix} 
a_n & a_{n-1} & \cdots & a_1 & a_0 & 0 & \cdots & 0 \\
0 & a_n & \cdots & a_2 & a_1	 & a_0 & \cdots & 0 \\
\vdots & \vdots & \ddots & \vdots & \vdots & \vdots & \ddots & \vdots\\
0 & 0 & \cdots & a_n & a_{n-1} & a_{n-2} & \cdots & a_0 \\
b_m & b_{m-1} & \cdots & b_1 & b_0 & 0 & \cdots & 0 \\
0 & b_m & \cdots & b_2 & b_1	 & b_0 & \cdots & 0 \\
\vdots & \vdots & \ddots & \vdots & \vdots & \vdots & \ddots & \vdots\\
0 & 0 & \cdots & b_m & b_{m-1} & b_{m-2} & \cdots & b_0 \\
   \end{pmatrix} 
&  \begin{array}{c} 
      \left. \rule{0pt}{12mm} \right\} \text{m} \\
      \left. \rule{0pt}{12mm} \right\} \text{n} 
   \end{array} 
\end{array}$$
\end{center}
\end{defn}

\begin{defn}
On appelle résultant de $A$ et $B$ le déterminant de la matrice de Sylvester de $A$ et $B$:
\begin{center} $Res(A,B) := det(Syl(A,B))$ \end{center}
\end{defn}

\begin{thm}[Admis]
Soient $A$ et $B \in \K[X]$, alors $Res(A,B) = 0$ ssi $A$ et $B$ ont un facteur commun non constant dans $\K[X]$.
\end{thm}

\begin{nota}
On notera $Res_Y(A,B)$ le résultant de deux polynôme en la variable $Y$ à coefficient dans $\K[X]$.
\end{nota}

Nous allons maintenant considérer $\alpha$ et $\beta$ deux éléments de $\jL$ algébriques sur $\K$. On notera respectivement leur polynômes minimaux $A(X)$ et $B(X)$ $\in \K[X]$, avec $deg(A) = n$ et $deg(B) = m$. L'objectif est de constuire un polynôme annulateur (et non forcément minimal !) de $\alpha + \beta$, $\alpha.\beta$ et $\alpha^{-1}$ afin de donner une preuve constructive du \textit{Théorème 2}.

\begin{prop}
La fermeture algébrique de $\K$ dans $\jL$ est munie d'une structure d'anneau; en effet
	\begin{enumerate}[label=\roman*)]
		\item Le polynôme $S(X) := Res_Y(A(Y),B(X-Y))$ est \textit{un} polynôme annulateur de $\alpha + \beta$.
		\item Le polynôme $P(X) := Res_Y(A(Y), X^m.B(\frac{X}{Y}))$ est \textit{un} polynôme annulateur de $\alpha.\beta$.
	\end{enumerate}
\end{prop}

\begin{proof} De simples calculs suffisent, remarquons que
	\begin{enumerate}[label=\roman*)]
		\item $S(\alpha + \beta) = Res_Y(A(Y),B(\alpha + \beta - Y))$. Or, $A(\alpha) = 0$ et $B (\alpha-\alpha + \beta) = B(\beta) = 0$. Donc les polynômes $A(Y)$ et $B (\alpha + \beta - Y) \in \K[Y]$ admettent $\alpha$ comme racine commune. De part le théorème précédent, on obtient que $S(\alpha + \beta) = Res_Y(A(Y),B(\alpha + \beta-Y)) = 0$, la conclusion s'ensuit. 
		\item De manière similaire, $P(\alpha.\beta) = Res_Y(A(Y), (\alpha.\beta)^m.B(\frac{\alpha.\beta}{Y}))$. Or, $A(\alpha) = 0$ et $(\alpha.\beta)^m.B(\frac{\alpha.\beta}{\alpha}) = (\alpha.\beta)^m.B(\beta) = 0$ Le terme $(\alpha.\beta)^m$ est nécessaire lorsque $\alpha = 0$. La conclusion s'ensuit.
	\end{enumerate}
\end{proof}
Et finalement :
\begin{prop}
La fermeture algébrique de $\K$ dans $\jL$ est munie d'une structure de corps; en effet le polynôme $P(X) := X^n.A(1/X)$ est un polynôme annulateur de $\alpha^{-1}$.
\end{prop}

\begin{proof}
Une fois de plus, un simple calcul suffit: \newline
$P(\alpha^{-1}) = ((\alpha^{-1})^n).A(\alpha^{-1}) = \alpha^{-n}.\displaystyle\sum_{i=0}^n (\frac{a_i}{\alpha^{-1}})^i = \alpha^{-n}.\sum_{i=0}^n \alpha^i.a_i = \alpha^{-n}.P(\alpha) = 0$
\end{proof}

\begin{exm}
Nous avons précedemment vu que le polynôme minimal de $\alpha = \sqrt2 + \sqrt3$ est $X^4 -10X^2 +1$. Retrouvons ce résultat grâce à la théorie des résultants. Soient $A$ et $B$ les polynômes minimaux de $\sqrt2$ et $\sqrt3$. Construisons $Syl(A(Y), B(Y-X))$. On a $B(Y-X) = (Y-X)^2 -3 = Y^2 + (-2X)Y + (X^2+3)$ d'où

\begin{center}
$Syl(A(Y), B(Y-X)) =$
$
   \begin{pmatrix} 
1 & 0 & -2 & 0 \\
0 & 1 & 0 & -2 \\
1 & -2X & X^2 +3 & 0 \\
0 & 1 & -2X & X^2 +3 \\
   \end{pmatrix} 
$
\end{center}

et donc

			\begin{center} $Res_Y((A(Y), B(Y-X))$ \end{center}
			\begin{align*} 
&= \begin{vmatrix} 1 & 0 & -2 & 0 \\ 0 & 1 & 0 & -2 \\ 1 & -2X & X^2 - 3 & 0 \\ 0 & 1 & -2X & X^2 -3 \\\end{vmatrix}  \\
&= Maple ! \\
&= X^4 -10X^2 +1 
			\end{align*} 
Ce qui correspond au polynôme minimal trouvé lors du précédent exemple. Nous avons ici obtenu un polynôme annulateur qui est \textit{le} polynôme minimal, mais ce ne sera pas toujorus le cas.
\end{exm}

\subsection{Brève disgression sur les corps finis}

On va ici s'interresser au cas particulier des corps finis.

\begin{nota}
On dénotera par $q := p^n$ la puissance n-ième d'un nombre premier $p$.
\end{nota}

\begin{prop}
Pour tout $p$ premier, il existe un corps fini à $p^n$ éléments, unique à isomorphisme près, qui sera noter $\Fq$.
\end{prop}

Contrairement au cas où le corps plancher est $\Q$, nous disposons d'algorithmes de construction de polynômes minimaux efficace. 

\begin{prop}
Soient $P$ un polynôme de degré $n$ à coefficients dans $\Fq$, et $\alpha$ une racine de $P$ dans $\Fqn$. Alors $P$ admet $n$ racines (distinctes !) dans $\Fqn$, qui ne sont autre que les $\alpha^{q^i}$, où $i$ décrit $\{1, ..., n-1\}$.
\end{prop}

\begin{defn}
Soit $\alpha$ un élement algébrique de degré $n$ sur $\Fq$. On appelle conjugués de $\alpha$ sur $\Fqn$ les racines de son polynôme minimal, i.e. les $\alpha^{q^i}$, où $i$ décrit $\{1, ..., n-1\}$.
\end{defn}

Il nous est maintenant facile de constuire le polynôme minimal (dans $\Fqn$ !) de $\alpha \in \Fq$.

\begin{algo}[Méthode des conjugués]
Soit $\alpha \in \Fqn$, on calcule les puissances successive de $\alpha^{q}$ jusqu'à trouver le plus petit entier $m$ tel que $\alpha^{q^m} = \alpha$. On obtient aisni que $\alpha$ est algébrique de degré $m$ et 
\begin{center} $Irr(\alpha, \Fq, X) = \displaystyle \prod_{i=0}^m (X-\alpha^{q^i})$. \end{center}
\end{algo}



%%%%%%%%%%%%%%%%%%%%%%%%%%%%%%%%%%%%%%%%%%%%%%%%%%%%%%%%%%%%%%%%%%%%%%%%%%%%%%%%%%%%%%%
\pagebreak
%%%%%%%%%%%%%%%%%%%%%%%%%%%%%%%%%%%%%%%%%%%%%%%%%%%%%%%%%%%%%%%%%%%%%%%%%%%%%%%%%%%%%%%
\section{Du code}




%%%%%%%%%%%%%%%%%%%%%%%%%%%%%%%%%%%%%%%%%%%%%%%%%%%%%%%%%%%%%%%%%%%%%%%%%%%%%%%%%%%%%%%
\pagebreak
%%%%%%%%%%%%%%%%%%%%%%%%%%%%%%%%%%%%%%%%%%%%%%%%%%%%%%%%%%%%%%%%%%%%%%%%%%%%%%%%%%%%%%%
\section{Des exemples}





%%%%%%%%%%%%%%%%%%%%%%%%%%%%%%%%%%%%%%%%%%%%%%%%%%%%%%%%%%%%%%%%%%%%%%%%%%%%%%%%%%%%%%%
\pagebreak
%%%%%%%%%%%%%%%%%%%%%%%%%%%%%%%%%%%%%%%%%%%%%%%%%%%%%%%%%%%%%%%%%%%%%%%%%%%%%%%%%%%%%%%

%\addcontentsline{toc}{part}{Références}

%\begin{thebibliography}{9}
	%\bibitem{these1}
	%Référence 1
%\end{thebibliography}

\end{document}
